%% ----------------------------------------
%%
%% NYU PhD thesis template.
%% Created by José Koiller 2007--2008.
%% Modified by Siddharth Krishna, 2019.
%%
%% ----------------------------------------


%% Use the first of the following lines during production to
%% easily spot "overfull boxes" in the output. Use the second
%% line for the final version.
%\documentclass[12pt,draft,letterpaper]{report}
\documentclass[12pt,oneside,letterpaper]{report}


% ----------------------------------------
% Macro to switch between draft version and final version
% ----------------------------------------

% Use or comment this to enable/disable draft version
% \def\draftversion{}
\newcommand{\draftfinal}[2]{\ifdefined\draftversion#1\else#2\fi}
\newcommand{\draftonly}[1]{\draftfinal{#1}{}}
\newcommand{\finalonly}[1]{\draftfinal{}{#1}}


% ----------------------------------------
% Thesis metadata
% ----------------------------------------

%% Replace the title, name, advisor name, graduation date and dedication below
%% with your own. Graduation months must be January, May or September.
\newcommand{\thesistitle}{On Complete Systems of Invariants for Ternary Biquadratic Forms}
\newcommand{\thesisauthor}{Amalie Emmy Noether}
\newcommand{\thesisadvisor}{Professor Paul Gordan}
\newcommand{\thesisdept}{Mathematics}
\newcommand{\gradmonth}{September}
\newcommand{\gradyear}{2019}
%% If you do not want a dedication, scroll down and comment out
%% the appropriate lines in this file.
\newcommand{\thesisdedication}{To my dog Weierstra\ss, with affection.}


% ----------------------------------------
% Layout and formatting
% ----------------------------------------

% Uncomment to get a big black box to spot "overfull hboxes"
% \setlength{\overfullrule}{5pt}


%% Page layout (customized to letter paper and NYU requirements):
\RequirePackage[margin=1in, includefoot, letterpaper]{geometry}


%% Color definitions:
\RequirePackage[prologue]{xcolor}
\definecolor[named]{ThesisBlue}{cmyk}{1,0.1,0,0.1}
\definecolor[named]{ThesisYellow}{cmyk}{0,0.16,1,0}
\definecolor[named]{ThesisOrange}{cmyk}{0,0.42,1,0.01}
\definecolor[named]{ThesisRed}{cmyk}{0,0.90,0.86,0}
\definecolor[named]{ThesisLightBlue}{cmyk}{0.49,0.01,0,0}
\definecolor[named]{ThesisGreen}{cmyk}{0.20,0,1,0.19}
\definecolor[named]{ThesisPurple}{cmyk}{0.55,1,0,0.15}
\definecolor[named]{ThesisDarkBlue}{cmyk}{1,0.58,0,0.21}

% School color found from university's graphic identity site:
% http://www.nyu.edu/employees/resources-and-services/media-and-communications/styleguide.html
\definecolor{SchoolColor}{rgb}{0.3412, 0.0235, 0.5490} % purple
\definecolor{chaptergrey}{rgb}{0.2600, 0.0200, 0.4600} % dialed back a little
\definecolor{midgrey}{rgb}{0.4, 0.4, 0.4}

\usepackage{hyperref}
\hypersetup{colorlinks,
  linkcolor=ThesisDarkBlue,
  citecolor=ThesisPurple,
  urlcolor=ThesisDarkBlue,
  filecolor=ThesisDarkBlue}


%% Captions of Figures, tables
\RequirePackage[labelfont={bf,sf,small,singlespacing},
                textfont={sf,small,singlespacing},
                % justification={justified,RaggedRight},
                % singlelinecheck=false,
                margin=0pt,
                figurewithin=chapter,
                tablewithin=chapter]{caption}

%% Chapter headings, captions
\usepackage{fix-cm}
\RequirePackage[raggedright,sc]{titlesec}
\definecolor{gray75}{gray}{0.75}
\newcommand{\hsp}{\hspace{20pt}}

\titleformat{\chapter}[hang]
{\Huge\sc}
{\textcolor{SchoolColor}{\thechapter}\hsp\textcolor{gray75}{|}\hsp}
{0pt}{\Huge\sc\raggedright}
% [\textcolor{gray75}{|}\hsp\textcolor{SchoolColor}{\thechapter}]


%% The following makes chapters and sections, but not subsections,
%% appear in the TOC (table of contents). Increase to 2 or 3 to
%% make subsections or subsubsections appear, respectively. It seems
%% to be usual to use the "1" setting, however.
\setcounter{tocdepth}{2}

%% Sectional units up to subsubsections are numbered. To number
%% subsections, but not subsubsections, decrease this counter to 2.
\setcounter{secnumdepth}{3}

%% Use the following commands, if desired, during production.
%% Comment them out for final version.
%\usepackage{layout} % defines the \layout command, see below
%\setlength{\hoffset}{-.75in} % creates a large right margin for notes and \showlabels

%% Controls spacing between lines (\doublespacing, \onehalfspacing, etc.):
\usepackage{setspace}

%% Use the line below for official NYU version, which requires
%% double line spacing. For all other uses, this is unnecessary,
%% so the line can be commented out.
\finalonly{
  \doublespacing % requires package setspace, invoked above
}

%% For generating sample text.
%% Can be removed when you've replaced all \lipsum commands with your text.
\usepackage{lipsum}


% ----------------------------------------
% Comments and TODOs:
% ----------------------------------------

% Uncomment this to remove all comments
\newcommand{\nocomments}{}

% Uncomment this to remove all TODOs
\newcommand{\notodos}{}

% Comments and TODOs
\newcommand{\fcomment}[2]{\ifdefined\nocomments{}\else\footnote{\textcolor{red}{#1:} #2}\fi}
\newcommand{\todo}[1]{\ifdefined\notodos{}\else\textcolor{red}{TODO\ifstrempty{#1}{}{: #1}}\fi}
\newcommand{\ftodo}[1]{\ifdefined\notodos{}\else\fcomment{TODO}{#1}\fi}

% Author comments:
\newcommand{\aen}[1]{\fcomment{Emmy}{#1}}


% ----------------------------------------
% User-specific packages and macros
% ----------------------------------------

%% This inputs your auxiliary file with \usepackage's and \newcommand's:
%% It is assumed that that file is called "defs.tex".
% ----------------------------------------
% Packages
% ----------------------------------------

% 
% Place here your \usepackage's. Some recommended packages are already included.
%

% Graphics:
\usepackage[final]{graphicx}
%\usepackage{graphicx} % use this line instead of the above to suppress graphics in draft copies
%\usepackage{graphpap} % \defines the \graphpaper command

% Uncomment this to indent first line of each section:
% \usepackage{indentfirst}

% Good AMS stuff:
\usepackage{amsthm} % facilities for theorem-like environments
\usepackage[tbtags]{amsmath} % a lot of good stuff!

% Fonts and symbols:
\usepackage{amsfonts}
\usepackage{amssymb}

% Set the fonts
\RequirePackage[T1]{fontenc}
\ifxetex
  \RequirePackage[tt=false]{libertine}
\else
  \RequirePackage[tt=false, type1=true]{libertine}
\fi
\RequirePackage[varqu]{zi4}
\RequirePackage[libertine]{newtxmath}


% For typesetting inference rules
\usepackage{mathpartir}
% \usepackage{pftools}  % A local package
\newcommand{\bmmax}{2}
\usepackage{bm}

% Formatting tools:
%\usepackage{relsize} % relative font size selection, provides commands \textsmalle, \textlarger
%\usepackage{xspace} % gentle spacing in macros, such as \newcommand{\acims}{\textsc{acim}s\xspace}

% Page formatting utility:
%\usepackage{geometry}

\usepackage{booktabs}   %% For formal tables:
                        %% http://ctan.org/pkg/booktabs
\usepackage[labelformat=simple]{subcaption} %% For complex figures with subfigures/subcaptions
                        %% http://ctan.org/pkg/subcaption
% Options to subcaption are to label and refer to subfigures as Fig 1(a) etc.
\renewcommand\thesubfigure{(\alph{subfigure})}

\usepackage[T1]{fontenc} % needed for scaling fancy fonts (?)
\usepackage[utf8]{inputenc} % not sure this is needed

\usepackage{amssymb}
%\usepackage[table]{xcolor}

% For code
\usepackage[final]{listings}
\lstset{mathescape=true}

% For code highlighting
% \usepackage{bold-extra}

% Tikz
\usepackage{tikz}
\usetikzlibrary{matrix,arrows,positioning,calc,fit,backgrounds}

% To control enum item labelling/numbering
\usepackage[shortlabels, inline]{enumitem}
% To give custom item labels and reference them
\makeatletter
\newcommand{\myitem}[1][]{
  \protected@edef\@currentlabel{#1}%
\item[#1]
}
\makeatother

% To stop aligned env swallowing up []s
\usepackage{mathtools}

% To use ifstrempty
\usepackage{etoolbox}

% For math mode tables
\usepackage{array}
% A text column in array
\newcolumntype{L}{>$l<$}

% For \llbracket and \rrbracket
\usepackage{stmaryrd}

% For dashed boxes
\usepackage{dashbox}

% For big separating conjunction
\usepackage{scalerel}

% For mathpar environment
\usepackage{mathpartir}

\usepackage{xspace}
\usepackage{multirow}

% To stop citations overflowing lines
\usepackage{breakcites}

% For citet command
\usepackage{natbib}
\setcitestyle{%
    authoryear,%
    open={[},close={]},citesep={;},%
    aysep={},yysep={,},%
    notesep={, }}
\let\cite\citep

%%
%% Place here your \newtheorem's:
%%

\theoremstyle{plain}
\newtheorem{theorem}{Theorem}[chapter]
\newtheorem{conjecture}[theorem]{Conjecture}
\newtheorem{proposition}[theorem]{Proposition}
\newtheorem{lemma}[theorem]{Lemma}
\newtheorem{corollary}[theorem]{Corollary}
\theoremstyle{definition}
\newtheorem{example}[theorem]{Example}
\newtheorem{definition}[theorem]{Definition}
\theoremstyle{plain}


% ----------------------------------------
% Generic definitions
% ----------------------------------------
% Required packages: listings, tikz

% A footnote without a marker
\newcommand\blfootnote[1]{%
  \begingroup
  \renewcommand\thefootnote{}\footnote{#1}%
  \addtocounter{footnote}{-1}%
  \endgroup
}

\renewcommand{\le}{\leqslant}
\renewcommand{\ge}{\geqslant}
% \renewcommand{\emptyset}{\ensuremath{\varnothing}}
% \newcommand{\ds}{\displaystyle}

% Math stuff
\newcommand{\R}{\ensuremath{\mathbb{R}}}
\newcommand{\Q}{\ensuremath{\mathbb{Q}}}
\newcommand{\Z}{\ensuremath{\mathbb{Z}}}
\newcommand{\N}{\ensuremath{\mathbb{N}}}
\newcommand{\T}{\ensuremath{\mathbb{T}}}
\newcommand{\C}{\ensuremath{\mathbb{C}}}
\newcommand{\eps}{\varepsilon}
\newcommand{\closure}[1]{\ensuremath{\overline{#1}}}
%\newcommand{\acim}{\textsc{acim}\xspace}
%\newcommand{\acims}{\textsc{acim}s\xspace}

\newcommand{\Land}{\bigwedge}
\newcommand{\Lor}{\bigvee}
\newcommand{\es}{\emptyset}
\newcommand{\incl}{\subseteq}
\newcommand{\impl}{\Rightarrow}
\renewcommand{\iff}{\Leftrightarrow}
\newcommand{\ra}{\rightarrow}
\newcommand{\sat}{\vDash}
\newcommand{\notsat}{\nvDash}
\newcommand{\proves}{\vdash}
\newcommand{\provesIff}{\mathrel{\dashv\vdash}}
\newcommand{\boolTrue}{\top}
\newcommand{\boolFalse}{\bot}

\newcommand{\dom}{\operatorname{\mathsf{dom}}}
\newcommand{\range}{\operatorname{\mathsf{rng}}}
\newcommand{\restrict}[2]{{#1}|_{#2}}
\newcommand{\pto}{\rightharpoonup}

\newcommand{\defeq}{\coloneqq}
\newcommand{\defiff}{\vcentcolon\iff}

\newcommand{\pipe}{\triangleright}

%% Caligraphic
\newcommand{\Aa}{{\mathcal{A}}}
\newcommand{\Bb}{{\mathcal{B}}}
\newcommand{\Cc}{{\mathcal{C}}}
\newcommand{\Dd}{{\mathcal{D}}}
\newcommand{\Ee}{{\mathcal{E}}}
\newcommand{\Ff}{{\mathcal{F}}}
\newcommand{\Gg}{{\mathcal{G}}}
\newcommand{\Hh}{{\mathcal{H}}}
\newcommand{\Ii}{{\mathcal{I}}}
\newcommand{\Jj}{{\mathcal{J}}}
\newcommand{\Kk}{{\mathcal{K}}}
\newcommand{\Ll}{{\mathcal{L}}}
\newcommand{\Mm}{{\mathcal{M}}}
\newcommand{\Nn}{{\mathcal{N}}}
\newcommand{\Oo}{{\mathcal{O}}}
\newcommand{\Pp}{{\mathcal{P}}}
\newcommand{\Qq}{{\mathcal{Q}}}
\newcommand{\Rr}{{\mathcal{R}}}
\newcommand{\Ss}{{\mathcal{S}}}
\newcommand{\Tt}{{\mathcal{T}}}
\newcommand{\Uu}{{\mathcal{U}}}
\newcommand{\Vv}{{\mathcal{V}}}
\newcommand{\Ww}{{\mathcal{W}}}
\newcommand{\Yy}{{\mathcal{Y}}}
\newcommand{\Zz}{{\mathcal{Z}}}

% Wrappers: Parens, brackets, etc
% \newcommand{\op}[1]{\operatorname{#1}}
\newcommand{\paren} [1] {\ensuremath{ \left( {#1} \right) }}
\newcommand{\bigparen} [1] {\ensuremath{ \Big( {#1} \Big) }}
% \newcommand{\bracket}[1]{\left[#1\right]}
\newcommand{\tuple}[1]{\ensuremath{\langle #1 \rangle}}
\newcommand{\abs}[1]{\ensuremath{\lvert #1 \rvert}}
% \newcommand{\set}[1]{\ensuremath{\left\{#1\right\}}}
\newcommand{\setcomp}[2]{\ensuremath{\left\{#1\;\middle|\;#2\right\}}}

% References
\newcommand{\refCh}[1]{Chapter~\ref{#1}}
\newcommand{\refSc}[1]{Section~\ref{#1}}
% \newcommand{\refSc}[1]{\S\ref{#1}}
\newcommand{\refFig}[1]{Figure~\ref{#1}}
\newcommand{\refDef}[1]{Definition~\ref{#1}}
\newcommand{\refLem}[1]{Lemma~\ref{#1}}
\newcommand{\refThm}[1]{Theorem~\ref{#1}}
\newcommand{\refAlg}[1]{Algorithm~\ref{#1}}
\newcommand{\refEx}[1]{Example~\ref{#1}}
\newcommand{\refCor}[1]{Corollary~\ref{#1}}
\newcommand{\refTab}[1]{Table~\ref{#1}}
\newcommand{\refEq}[1]{\ensuremath{(\ref{#1})}}
\newcommand{\refRule}[1]{(\ref{#1})}
\newcommand{\refApp}[1]{Appendix~\ref{#1}}

\newcommand{\tool}[1]{\textsf{#1}}
\newcommand{\code}[1]{\textnormal{\small\texttt{#1}}}
% \newcommand{\code}[1]{\text{\lstinline{#1}}}

% TODO have macros for \forall and \exists

\newcommand{\tick}{\ensuremath{\checkmark}}
\newcommand{\cross}{\text{\sffamily X}}


% ----------------------------------------
% Paper specific macros & commands
% ----------------------------------------


% Put your definitions here


%%% Local Variables:
%%% mode: latex
%%% TeX-master: "thesis"
%%% End:



% ----------------------------------------
% Document header
% ----------------------------------------

%% Cross-referencing utilities. Use one or the other--whichever you prefer--
%% but comment out both lines for final version.
%\usepackage{showlabels}
%\usepackage{showkeys}

\begin{document}
%% Produces a test "layout" page, for "debugging" purposes only.
%% Comment out for final version.
%\layout % requires package layout (see above, on this same file)


%%%%%% Title page %%%%%%%%%%%
%% Sets page numbering to "roman style" i, ii, iii, iv, etc:
\pagenumbering{roman}
%
%% No numbering in the title page:
\thispagestyle{empty}
%
\vspace*{25pt}
\begin{center}
  {\Large
    \begin{doublespace}
      {\textcolor{SchoolColor}{\textsc{\thesistitle}}}
    \end{doublespace}
  }
  \vspace{.7in}

  by
  \vspace{.7in}

  \thesisauthor
  \vfill

  \begin{doublespace}
    \textsc{
    A dissertation submitted in partial fulfillment\\
    of the requirements for the degree of\\
    Doctor of Philosophy\\
    Department of \thesisdept\\
    New York University\\
    \gradmonth, \gradyear}
  \end{doublespace}
\end{center}
\vfill

\noindent\makebox[\textwidth]{\hfill\makebox[2.5in]{\hrulefill}}\\
\makebox[\textwidth]{\hfill\makebox[2.5in]{\hfill\thesisadvisor}}

\newpage


%%%%%%%%%%%%% Copyright page %%%%%%%%%%%%%%%%%%
\thispagestyle{empty}
\vspace*{25pt}
\begin{center}
  \scshape \noindent \small \copyright \  \small  \thesisauthor \\
  all rights reserved, \gradyear
\end{center}
\vspace*{0in}
\newpage


%%%%%%%%%%%%%% Dedication %%%%%%%%%%%%%%%%%
%% Comment out the following lines if you do not want to dedicate
%% this to anyone...
\cleardoublepage
\phantomsection
\addcontentsline{toc}{chapter}{Dedication}
\vspace*{\fill}
\begin{center}
  \thesisdedication
\end{center}
\vfill
\newpage


%%%%%%%%%%%%%% Acknowledgements %%%%%%%%%%%%
%% Comment out the following lines if you do not want to acknowledge
%% anyone's help...
\chapter*{Acknowledgements}
\addcontentsline{toc}{chapter}{Acknowledgments}

% \input{acknowledge}
\lipsum[1-2]

\newpage


%%%% Abstract %%%%%%%%%%%%%%%%%%
\chapter*{Abstract}
\addcontentsline{toc}{chapter}{Abstract}

% Replace this with your abstract:
\lipsum[3-4]


%%% Local Variables:
%%% mode: latex
%%% TeX-master: "thesis"
%%% End:


\newpage


%%%% Table of Contents %%%%%%%%%%%%
\tableofcontents


%%%%% List of Figures %%%%%%%%%%%%%
%% Comment out the following two lines if your thesis does not
%% contain any figures. The list of figures contains only
%% those figures included within the "figure" environment.
\cleardoublepage
\phantomsection
\addcontentsline{toc}{chapter}{List of Figures}
\listoffigures
\newpage


%%%%% List of Tables %%%%%%%%%%%%%
%% Comment out the following two lines if your thesis does not
%% contain any tables. The list of tables contains only
%% those tables included within the "table" environment.
\cleardoublepage
\phantomsection
\addcontentsline{toc}{chapter}{List of Tables}
\listoftables
\newpage


%%%%% Body of thesis starts %%%%%%%%%%%%
\pagenumbering{arabic} % switches page numbering to arabic: 1, 2, 3, etc.


% ----------------------------------------
% Body of Thesis
% ----------------------------------------

\chapter{Introduction}
\label{chp-introduction}

% \input{introduction}

% Sample content:
\lipsum[1-2]

\begin{figure}
  \centering
  \begin{tikzpicture}[scale=3]
    \draw[step=.5cm, gray, very thin] (-1.2,-1.2) grid (1.2,1.2); 
    \filldraw[fill=green!20,draw=green!50!black] (0,0) -- (3mm,0mm) arc (0:30:3mm) -- cycle; 
    \draw[->] (-1.25,0) -- (1.25,0) coordinate (x axis);
    \draw[->] (0,-1.25) -- (0,1.25) coordinate (y axis);
    \draw (0,0) circle (1cm);
    \draw[very thick,red] (30:1cm) -- node[left,fill=white] {$\sin \alpha$} (30:1cm |- x axis);
    \draw[very thick,blue] (30:1cm |- x axis) -- node[below=2pt,fill=white] {$\cos \alpha$} (0,0);
    \draw (0,0) -- (30:1cm);
    \foreach \x/\xtext in {-1, -0.5/-\frac{1}{2}, 1} 
    \draw (\x cm,1pt) -- (\x cm,-1pt) node[anchor=north,fill=white] {$\xtext$};
    \foreach \y/\ytext in {-1, -0.5/-\frac{1}{2}, 0.5/\frac{1}{2}, 1} 
    \draw (1pt,\y cm) -- (-1pt,\y cm) node[anchor=east,fill=white] {$\ytext$};
  \end{tikzpicture}
  \caption{A pictorial view of \refThm{thm-pythagorean}.}
  \label{fig-pythagorean}
\end{figure}

\begin{definition}
  A function $f$ is said to be \emph{continuous} if its derivative exists at every point.
\end{definition}

\begin{lemma}
  Let $f$ be a function whose derivative exists in every point, then $f$ is 
  a continuous function.
\end{lemma}

\begin{theorem}[Pythagorean theorem]
  \label{thm-pythagorean}
  This is a theorema about right triangles and can be summarised in the next 
  equation 
  \[ x^2 + y^2 = z^2 \]
\end{theorem}
\begin{proof}
  I have discovered a truly marvelous proof of this, which this margin is too narrow to contain.
\end{proof}

And a consequence of \refThm{thm-pythagorean} is the statement in the next 
corollary~\cite{lamport94}.

\begin{corollary}
  There's no right rectangle whose sides measure 3cm, 4cm, and 6cm.
\end{corollary}

\lipsum[3-5]

\begin{table}
  \centering
  \caption{Predicted final standings of Group B.}
  \begin{tabular}{l*{6}{c}r}
    Team              & P & W & D & L & F  & A & Pts \\
    \hline
    Manchester United & 6 & 4 & 0 & 2 & 10 & 5 & 12  \\
    Celtic            & 6 & 3 & 0 & 3 &  8 & 9 &  9  \\
    Benfica           & 6 & 2 & 1 & 3 &  7 & 8 &  7  \\
    FC Copenhagen     & 6 & 2 & 1 & 3 &  5 & 8 &  7  \\
  \end{tabular}
  \label{tab-forecast}
\end{table}

\lipsum[5-7]

\chapter{Preliminaries}
\label{chp-preliminaries}

% \input{preliminaries}
\lipsum

\chapter{Proof of the Reimann Hypothesis}
\label{chp-proof}

% \input{proof}
\lipsum

\chapter{Conclusion}
\label{chp-conclusion}

% \input{conclusion}
\lipsum


%% If your thesis has different "Parts", use commands such as the following:
%\part{First Part\label{part:one}}%
% \input{chap1}
%\input{chap2} % further chapters -- change file names to meaningful things...
%\input{chap3}
%\part{Second Part\label{part:two}}%
%\input{chap4}
%\input{chap5}
%\input{chap6}


%%%%% Appendices start %%%%%%%%%%%%%%%%
%% Comment out the following if your thesis has no appendix

\appendix

\chapter{Appendix}

% \input{appendix}
\lipsum

%% Note: If your thesis has more than one appendix, NYU requires a "list of
%% appendices" page before the body of the thesis. I don't provide the tools
%% to create that here, so you're on your own for that one... Sorry.


%%%% Input bibliography file %%%%%%%%%%%%%%%
%% For computer science dissertations, I'd recommend using the bibly package
%% to automatically create the .bib file from your citations:
%% https://github.com/michael-emmi/bibly

\cleardoublepage
\phantomsection
\bibliographystyle{apalike}
\addcontentsline{toc}{chapter}{Bibliography}

% \bibliography{dblp,references}

% The following is just for the sample template,
% I'd recommend deleting this and using the \bibliography command above
\begin{thebibliography}{99}
\bibitem[Lamport, 1994]{lamport94}
  Leslie Lamport,
  \textit{\LaTeX: a document preparation system},
  Addison Wesley, Massachusetts,
  2nd edition,
  1994.
\end{thebibliography}


\end{document}

%%% Local Variables:
%%% mode: latex
%%% TeX-master: t
%%% End:
