% ----------------------------------------
% Packages
% ----------------------------------------

% 
% Place here your \usepackage's. Some recommended packages are already included.
%

% Graphics:
\usepackage[final]{graphicx}
%\usepackage{graphicx} % use this line instead of the above to suppress graphics in draft copies
%\usepackage{graphpap} % \defines the \graphpaper command

% Uncomment this to indent first line of each section:
% \usepackage{indentfirst}

% Good AMS stuff:
\usepackage{amsthm} % facilities for theorem-like environments
\usepackage[tbtags]{amsmath} % a lot of good stuff!

% Fonts and symbols:
\usepackage{amsfonts}
\usepackage{amssymb}

% Set the fonts
\RequirePackage[T1]{fontenc}
\ifxetex
  \RequirePackage[tt=false]{libertine}
\else
  \RequirePackage[tt=false, type1=true]{libertine}
\fi
\RequirePackage[varqu]{zi4}
\RequirePackage[libertine]{newtxmath}


% For typesetting inference rules
\usepackage{mathpartir}
% \usepackage{pftools}  % A local package
\newcommand{\bmmax}{2}
\usepackage{bm}

% Formatting tools:
%\usepackage{relsize} % relative font size selection, provides commands \textsmalle, \textlarger
%\usepackage{xspace} % gentle spacing in macros, such as \newcommand{\acims}{\textsc{acim}s\xspace}

% Page formatting utility:
%\usepackage{geometry}

\usepackage{booktabs}   %% For formal tables:
                        %% http://ctan.org/pkg/booktabs
\usepackage[labelformat=simple]{subcaption} %% For complex figures with subfigures/subcaptions
                        %% http://ctan.org/pkg/subcaption
% Options to subcaption are to label and refer to subfigures as Fig 1(a) etc.
\renewcommand\thesubfigure{(\alph{subfigure})}

\usepackage[T1]{fontenc} % needed for scaling fancy fonts (?)
\usepackage[utf8]{inputenc} % not sure this is needed

\usepackage{amssymb}
%\usepackage[table]{xcolor}

% For code
\usepackage[final]{listings}
\lstset{mathescape=true}

% For code highlighting
% \usepackage{bold-extra}

% Tikz
\usepackage{tikz}
\usetikzlibrary{matrix,arrows,positioning,calc,fit,backgrounds}

% To control enum item labelling/numbering
\usepackage[shortlabels, inline]{enumitem}
% To give custom item labels and reference them
\makeatletter
\newcommand{\myitem}[1][]{
  \protected@edef\@currentlabel{#1}%
\item[#1]
}
\makeatother

% To stop aligned env swallowing up []s
\usepackage{mathtools}

% To use ifstrempty
\usepackage{etoolbox}

% For math mode tables
\usepackage{array}
% A text column in array
\newcolumntype{L}{>$l<$}

% For \llbracket and \rrbracket
\usepackage{stmaryrd}

% For dashed boxes
\usepackage{dashbox}

% For big separating conjunction
\usepackage{scalerel}

% For mathpar environment
\usepackage{mathpartir}

\usepackage{xspace}
\usepackage{multirow}

% To stop citations overflowing lines
\usepackage{breakcites}

% For citet command
\usepackage{natbib}
\setcitestyle{%
    authoryear,%
    open={[},close={]},citesep={;},%
    aysep={},yysep={,},%
    notesep={, }}
\let\cite\citep

%%
%% Place here your \newtheorem's:
%%

\theoremstyle{plain}
\newtheorem{theorem}{Theorem}[chapter]
\newtheorem{conjecture}[theorem]{Conjecture}
\newtheorem{proposition}[theorem]{Proposition}
\newtheorem{lemma}[theorem]{Lemma}
\newtheorem{corollary}[theorem]{Corollary}
\theoremstyle{definition}
\newtheorem{example}[theorem]{Example}
\newtheorem{definition}[theorem]{Definition}
\theoremstyle{plain}


% ----------------------------------------
% Generic definitions
% ----------------------------------------
% Required packages: listings, tikz

% A footnote without a marker
\newcommand\blfootnote[1]{%
  \begingroup
  \renewcommand\thefootnote{}\footnote{#1}%
  \addtocounter{footnote}{-1}%
  \endgroup
}

\renewcommand{\le}{\leqslant}
\renewcommand{\ge}{\geqslant}
% \renewcommand{\emptyset}{\ensuremath{\varnothing}}
% \newcommand{\ds}{\displaystyle}

% Math stuff
\newcommand{\R}{\ensuremath{\mathbb{R}}}
\newcommand{\Q}{\ensuremath{\mathbb{Q}}}
\newcommand{\Z}{\ensuremath{\mathbb{Z}}}
\newcommand{\N}{\ensuremath{\mathbb{N}}}
\newcommand{\T}{\ensuremath{\mathbb{T}}}
\newcommand{\C}{\ensuremath{\mathbb{C}}}
\newcommand{\eps}{\varepsilon}
\newcommand{\closure}[1]{\ensuremath{\overline{#1}}}
%\newcommand{\acim}{\textsc{acim}\xspace}
%\newcommand{\acims}{\textsc{acim}s\xspace}

\newcommand{\Land}{\bigwedge}
\newcommand{\Lor}{\bigvee}
\newcommand{\es}{\emptyset}
\newcommand{\incl}{\subseteq}
\newcommand{\impl}{\Rightarrow}
\renewcommand{\iff}{\Leftrightarrow}
\newcommand{\ra}{\rightarrow}
\newcommand{\sat}{\vDash}
\newcommand{\notsat}{\nvDash}
\newcommand{\proves}{\vdash}
\newcommand{\provesIff}{\mathrel{\dashv\vdash}}
\newcommand{\boolTrue}{\top}
\newcommand{\boolFalse}{\bot}

\newcommand{\dom}{\operatorname{\mathsf{dom}}}
\newcommand{\range}{\operatorname{\mathsf{rng}}}
\newcommand{\restrict}[2]{{#1}|_{#2}}
\newcommand{\pto}{\rightharpoonup}

\newcommand{\defeq}{\coloneqq}
\newcommand{\defiff}{\vcentcolon\iff}

\newcommand{\pipe}{\triangleright}

%% Caligraphic
\newcommand{\Aa}{{\mathcal{A}}}
\newcommand{\Bb}{{\mathcal{B}}}
\newcommand{\Cc}{{\mathcal{C}}}
\newcommand{\Dd}{{\mathcal{D}}}
\newcommand{\Ee}{{\mathcal{E}}}
\newcommand{\Ff}{{\mathcal{F}}}
\newcommand{\Gg}{{\mathcal{G}}}
\newcommand{\Hh}{{\mathcal{H}}}
\newcommand{\Ii}{{\mathcal{I}}}
\newcommand{\Jj}{{\mathcal{J}}}
\newcommand{\Kk}{{\mathcal{K}}}
\newcommand{\Ll}{{\mathcal{L}}}
\newcommand{\Mm}{{\mathcal{M}}}
\newcommand{\Nn}{{\mathcal{N}}}
\newcommand{\Oo}{{\mathcal{O}}}
\newcommand{\Pp}{{\mathcal{P}}}
\newcommand{\Qq}{{\mathcal{Q}}}
\newcommand{\Rr}{{\mathcal{R}}}
\newcommand{\Ss}{{\mathcal{S}}}
\newcommand{\Tt}{{\mathcal{T}}}
\newcommand{\Uu}{{\mathcal{U}}}
\newcommand{\Vv}{{\mathcal{V}}}
\newcommand{\Ww}{{\mathcal{W}}}
\newcommand{\Yy}{{\mathcal{Y}}}
\newcommand{\Zz}{{\mathcal{Z}}}

% Wrappers: Parens, brackets, etc
% \newcommand{\op}[1]{\operatorname{#1}}
\newcommand{\paren} [1] {\ensuremath{ \left( {#1} \right) }}
\newcommand{\bigparen} [1] {\ensuremath{ \Big( {#1} \Big) }}
% \newcommand{\bracket}[1]{\left[#1\right]}
\newcommand{\tuple}[1]{\ensuremath{\langle #1 \rangle}}
\newcommand{\abs}[1]{\ensuremath{\lvert #1 \rvert}}
% \newcommand{\set}[1]{\ensuremath{\left\{#1\right\}}}
\newcommand{\setcomp}[2]{\ensuremath{\left\{#1\;\middle|\;#2\right\}}}

% References
\newcommand{\refCh}[1]{Chapter~\ref{#1}}
\newcommand{\refSc}[1]{Section~\ref{#1}}
% \newcommand{\refSc}[1]{\S\ref{#1}}
\newcommand{\refFig}[1]{Figure~\ref{#1}}
\newcommand{\refDef}[1]{Definition~\ref{#1}}
\newcommand{\refLem}[1]{Lemma~\ref{#1}}
\newcommand{\refThm}[1]{Theorem~\ref{#1}}
\newcommand{\refAlg}[1]{Algorithm~\ref{#1}}
\newcommand{\refEx}[1]{Example~\ref{#1}}
\newcommand{\refCor}[1]{Corollary~\ref{#1}}
\newcommand{\refTab}[1]{Table~\ref{#1}}
\newcommand{\refEq}[1]{\ensuremath{(\ref{#1})}}
\newcommand{\refRule}[1]{(\ref{#1})}
\newcommand{\refApp}[1]{Appendix~\ref{#1}}

\newcommand{\tool}[1]{\textsf{#1}}
\newcommand{\code}[1]{\textnormal{\small\texttt{#1}}}
% \newcommand{\code}[1]{\text{\lstinline{#1}}}

% TODO have macros for \forall and \exists

\newcommand{\tick}{\ensuremath{\checkmark}}
\newcommand{\cross}{\text{\sffamily X}}


% ----------------------------------------
% Paper specific macros & commands
% ----------------------------------------


% Put your definitions here


%%% Local Variables:
%%% mode: latex
%%% TeX-master: "thesis"
%%% End:
